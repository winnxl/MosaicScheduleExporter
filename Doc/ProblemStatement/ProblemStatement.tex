\documentclass{article}

\usepackage{tabularx}
\usepackage{booktabs}

\title{SE 3XA3: Problem Statement\\MAC Schedule Importer}

\author{Team 12%, Team Name
		\\ Cassandra Nicolak, nicolace
		\\ Michelle Leung, leungm16
		\\ Winnie Liang, liangw15
}

\date{}

%\input{../Comments}

\begin{document}

\begin{table}[hp]
\caption{Revision History} \label{TblRevisionHistory}
\begin{tabularx}{\textwidth}{llX}
\toprule
\textbf{Date} & \textbf{Developer(s)} & \textbf{Change}\\
\midrule
09/20/2018 & Cassandra Nicolak, Michelle Leung, Winnie Liang & Rev 0\\
%Date2 & Name(s) & Description of changes\\
%... & ... & ...\\
\bottomrule
\end{tabularx}
\end{table}

\newpage

\maketitle

\section{Overview}
\hspace{5mm}Time management and organization is essential to success while attending a post-secondary institution. Having a detailed organizer is one method students use to keep track of their many responsibilities while attending University. Students want a way to have this information easily accessible and consolidated into one platform as it is tedious and inefficient to check multiple websites and applications. Many McMaster students use devices such as smartphones and laptops to plan their events and schedules. Hence, a tool that will allow students to have an increased ease of access to their schedules that incorporates both academic and personal responsibilities is invaluable.


\section{Motivation}
\hspace{5mm}Mosaic, McMaster's administrative and informative website, creates a schedule within its current website that students can use to visualize their day. As users of Mosaic know, it undergoes maintenance during which the site is available. Since the schedule is within the website, it is separate from other schedules for events and appointments an individual might have. \\

Mosaic has a tool that allows users to import their timetable into their Google calendar; however, this tool has many limitations. For instance, the tool is not supported on all browsers (Microsoft Edge and Internet Explorer) and users must wait a day after schedule changes before they can use the tool. Mosaic isn’t too mobile friendly either as frames cut off most of the schedule. Manually entering in this information into your Google calendar is also tedious and not an efficient use of time. Since time is such a valuable resource as a student, a more effective method of organizing this information would be useful.


\section{Objectives and Scope}
\hspace{5mm}The stakeholders for this software are McMaster University students, staff members, as well as current (Team 12) and future software developers. Other stakeholders include Mosaic and organizers such as Google Calendar.\\

UMD Google Calendar Schedule Importer is open-source Google extension that imports the class schedule for students at the University of Maryland into Google Calendar. Also, as mentioned above, Mosaic’s import tool is very limited. The scope of this project will be to re-implement the Google Chrome extension into a  Desktop application tailored towards McMaster University students. This will make it easier for users to import their schedules from Mosaic and into their Google calendar which they can access easily with their mobile phone or through a browser.


%\wss{comment}

%\ds{comment}

%\mj{comment}

%\cm{comment}

%\mh{comment}

\end{document}