\documentclass{article}

\usepackage{booktabs}
\usepackage{tabularx}
\usepackage{hyperref}
\usepackage{color}

\title{SE 3XA3: Development Plan\\MAC Schedule Importer}

\author{Team 12, Team 0C
		\\ Cassandra Nicolak, nicolace
		\\ Michelle Leung, leungm16
		\\ Winnie Liang, liangw15
}

\date{}

%\input{../Comments}

\begin{document}

\begin{table}[hp]
\caption{Revision History} \label{TblRevisionHistory}
\begin{tabularx}{\textwidth}{llX}
\toprule
\textbf{Date} & \textbf{Developer(s)} & \textbf{Change}\\
\midrule
2018/09/28 & Cassandra Nicolak, Michelle Leung, Winnie Liang & Rev0\\
\color{blue}
2018/12/05 & \color{blue} Cassandra Nicolak, Michelle Leung, Winnie Liang & \textcolor{blue}{Rev 1}\\
%Date2 & Name(s) & Description of changes\\
%... & ... & ...\\
\bottomrule
\end{tabularx}
\end{table}

\newpage

\maketitle

Our team will be being redesigning an existing Chrome extension that allows a user to import their school schedule (from their institution's website) into their Google calendar.


\section{Team Meeting Plan}
\subsection{Logistics}
\hspace{5mm}We will have both in-person and online meetings throughout the term. In-person meetings will be held once a week an hour before our first lab of the week (Tuesdays at 17:45 - 18:45) and online meetings will vary depending on necessity. Any online meetings that will be needed for the week will be decided during the in-person meeting on Tuesdays. \\

Online meetings will be held through Skype. The location for the in-person meeting will be posted in team's Facebook group the day before. A more consistent meeting place will be decided after the second in-person meeting and will be added in the next revision of this document.

\subsection{Meeting Roles}
\hspace{5mm}Meeting roles will vary from week-to-week depending on the tasks and deadlines. The agenda for the following week's in-person meeting will be discussed in the last 20 minutes of the current meeting. This will include who will be the next week's chair and scribe.\\

\begin{itemize}
\item Meeting Chair: Facilitates the meeting for the week and leads the meeting agenda. The chair will be determined based on the topic.
\item Scribe: Records the minutes during the meeting (fills in agenda template).
\item Planner: Updates the Gantt chart if needed.
\end{itemize}

\subsection{Agenda Outline}
\begin{itemize}
\item Meeting Topic
\item Date/Time/Location
\item Meeting Chair
\item Attendees
\item Meeting Objective
\item Status of tasks assigned from last meeting
\item Obstacles and issues that need to be addressed
\item Tasks to be completed for next meeting
\item Objective for next meeting
\item Meeting Notes

\end{itemize}
\section{Team Communication Plan}
\hspace{5mm}The team's Facebook group chat will be used for general discussion and on-going issues. When unexpected issues arise and an emergency meeting is needed, this will be communicated through the team's Facebook group. If member(s) are unreachable through Facebook, they will be contacted through text and phone call.
Skype has a screen sharing feature and will be used when all members are required for troubleshooting. Skype will also be the main method of communication for online meetings.
A team Discord or Slack will be created before the next revision of this document and will have the following different channels for specific tasks:
\begin{itemize}
\item General
\item Git
\item Documentation
\item Programming - Different channels will be made for sub-tasks.
\item Troubleshooting
\item Testing
\item Meeting Minutes
\item Holidays/Vacation Notice

\end{itemize}


\section{Team Member Roles}
\hspace{5mm}All team members are experienced with programming in Python 3, version control with Git and LaTeX. Michelle is more experienced with documentation requirements so she will take on the primary role of Documentation Lead. Winnie will focus on the technical aspects of the project and will take on the primary role of Software Lead. Cassandra will focus on the overall structure and flow of tasks and take on the primary role of Project Manager.\\

Secondary roles will be assigned on a weekly basis and reflected in the team's Gantt chart.

\subsection{Primary Role}
\begin{itemize}
\item Software Lead: Ensures that all responsibilities and requirements for programming and software are met. 
\item Documentation Lead: Ensures that all responsibilities and requirements for documentation are met. Also the Gantt Chart Facilitator.
\item Project Manager: Responsible for the time management of tasks and deadlines.
\end{itemize}

\subsection{Secondary Role}
\begin{itemize}
\item Writer: Writes a rough draft of the document (or section of document) into proper sentences. Entire group will share the task of brainstorming ideas in Google Docs.
\item Editor: Edits the rough draft.
\item Scribe: Responsible for meeting minutes. 
\item Programmer: Implements the software design for the assigned function/task.
\item Tester: Develops and administers test cases.
\end{itemize}

\section{Git Workflow Plan}
\begin{itemize}
\item Master branch should always be functional.
\item Working branches will be made according to the needs and areas of specialization:
\begin{itemize}
\item Front-End
\item Back-End
\item Documentation
\end{itemize}

\item Branches are to be synced at least one a week in-person. 
\item Tags will be used for deliverables.

\item Milestones are used to mark important deadlines on the Gantt chart. They are also used as a guide to ensure that our project is progressing on time.

\end{itemize}

\section{Proof of Concept Demonstration Plan}

The team will prepare for the proof of concept demonstration during the next meeting. We will:

\begin{itemize}
\item Design a basic version that illustrates the primary function of the program.
\item Describe what we will demonstrate to show that risks can
be overcome and that the project is feasible.
\item Find libraries and discuss what changes needs to be made from the source code.
\item Identify the scope and divide up steps for the final implementation.
\item Address potential obstacles that could be encountered during development.
\item Create a test plan and identify possible use cases.
\item Identify what platforms we wish to target.

\end{itemize}

\section{Technology}
\begin{itemize}
\item Languages of source code: JavaScript, HTML, CSS
\item Programming language for port: Python 3
\item IDE: Visual Studio Code, Python IDLE, Pycharm
\item Document generation: LaTeX through TeXworks.
\item Testing framework: Will be  decided before developing test cases.

\end{itemize}
\section{Coding Style}
Our coding style will follow the following style guide:\color{blue} 
\href{https://google.github.io/styleguide/pyguide.html}{ Google Python Style Guide} 

\color{black}
\section{Project Schedule}

Please refer to the link to find the current Gantt Chart:
\color{blue}
\href{https://gitlab.cas.mcmaster.ca/liangw15/3XA3Project/blob/master/Doc/DevelopmentPlan/Group12_Gantt03.pdf}{ Gantt03}
%\href{run:C:/Users/Michelle/Documents/M3/Year 3/3XA3/Group12_Gantt01.pdf}{FILE}
\color{black}

\section{Project Review}
\color{blue}
\subsection{Reflection Overview}
\hspace{5mm}
After completing the project, the team gained a deeper understanding of the documentation and software design process. Through the completion of each step of the project, the group was able to experience the various challenges and how to overcome such obstacles in software design. Moreover, each team member acquired invaluable teamwork, organization and communication skills that are critical for future success in the industry. Thus, this project is an important milestone and learning experience for each individual of Team 0C.

\subsection{Project Strengths}
\hspace{5mm} Throughout the project, the team exhibited excellent teamwork and communication skills. All the deadlines were met and various tasks were divided accordingly. \\


In particular, the team had a balanced group dynamic. One team member with a deeper understanding of the various documentations was in charge of the documentation. On the other hand, another team member excelled in software development and design and was in charge of the implementation of the project. Furthermore, the team leader was in charge of supervision and balancing the workload of the entire group. Thus, each member of the team was given tasks according the each individual's strength and weakness. \\

Moreover, the team demonstrated great communication as a team. Various methods of communications were used for different types of communication including group chats such as Facebook Messenger, Skype, Discord and Google Drive. Online meetings were held via Skype to allow screen sharing and direct communication. Facebook Messenger was used as a means of quick and casual communication between the group. As well, Google Drive was used to share all the required documentations and external resources to the team. Additionally, Discord was used to make changes, announcements and discussion on the implementation of the application.\\


Another strength the project displayed was the decomposition and design in implementing the program. The team decomposed the program into different modules to ensure that only one secret was used in each module. Furthermore, the design of the module connections was implemented with high cohesion and low coupling principles.
\subsection{Project Challenges}
\hspace{5mm}There were many challenges that we faced over the course of this project. \\


One challenge was implementing the web-scraping framework. Since no one in the team had prior knowledge and experience in this field, the team had to learn web-scraping. As a result of the lack of domain background, there was multiple problems incorporating the web-scraping function of the application.\\


Another challenge was converting the information retrieved from Mosaic into usable information for Google Calendars. For instance, there were issues in modifying the data into the correct time zones as well as an overlapping problem for the Monday schedule. Furthermore, there was a problem in converting the 24 hour time period into a 12 hour time period that the Google API accepted.\\


Finally, the task of converting the Python program into an executable application was an unexpected obstacle we faced in this program. Initially, the team used the software, PyInstaller, to implement the executable. However, multiple import errors occurred because PyInstaller was unable to include all the necessary files and modules that Scrapy needed to function. Hence, after trying different software, the conversion was completed through Cx\_Freeze. Consequently, the original plan of converting the program into a single executable file was changed into a single folder.

\subsection{Future Improvements}
\hspace{5mm} Despite the superb work done for this project, there are many areas that can be improved upon for the future. \\

For instance, the frequency and duration for team meetings can be adjusted to accommodate each team member's schedule and the weight load for that particular week. During the midterm season,  face-to-face team meetings can be changed into online meetings to accommodate the busy schedules of each member. As well, rather than infrequent and long team meetings, short but frequent meetings can be incorporated to allow flexibility in the team meeting times. \\


Another area of improvement that this project can have is time management. Although the team finish all the required tasks before every deadline, the work was often completed last-minute. Better time management and including self-checkpoints can be implemented in future projects to ensure the work is completed with time allocated for possible changes. 

\subsection{Final Comments}
\hspace{5mm}Overall, the project was an enjoyable and educational process that enabled each member of Team0C to grow in their experience and knowledge as software engineers.

\end{document}

