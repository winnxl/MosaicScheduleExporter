\documentclass[12pt, titlepage]{article}

\usepackage{booktabs}
\usepackage{tabularx}
\usepackage{hyperref}
\hypersetup{
    colorlinks,
    citecolor=black,
    filecolor=black,
    linkcolor=red,
    urlcolor=blue
}
\usepackage[round]{natbib}

\title{SE 3XA3: Test Plan\\MAC Schedule Importer}

\author{Team 12, Team 0C
		\\ Cassandra Nicolak, nicolace
		\\ Michelle Leung, leungm16
		\\ Winnie Liang, liangw15
}

\date{\today}

%\input{../Comments}

\begin{document}

\maketitle

\pagenumbering{roman}
\tableofcontents
\listoftables
\listoffigures

\begin{table}[bp]
\caption{\bf Revision History}
\begin{tabularx}{\textwidth}{p{3cm}p{2cm}X}
\toprule {\bf Date} & {\bf Version} & {\bf Notes}\\
\midrule
2018-10-25 & 1.0 & Rough Draft\\
2018-10-26 & 1.1 & Final Draft\\
\bottomrule
\end{tabularx}
\end{table}

\newpage

\pagenumbering{arabic}

%This document ...

\section{General Information}

\subsection{Purpose}
The purpose of the test plan is to build confidence in the implementation correctness for the project. 

\subsection{Scope}
The scope of the test plan is to provide a basic testing platform for the correctness and functionality of the software. The test plan includes a description of the testing procedures, tools and test cases that will be implemented to verify and validate the software for this project.
\subsection{Acronyms, Abbreviations, and Symbols}
	
\begin{table}[hbp]
\caption{\textbf{Table of Abbreviations}} \label{Table}

\begin{tabularx}{\textwidth}{p{3cm}X}
\toprule
\textbf{Abbreviation} & \textbf{Definition} \\
\midrule
CSV & Comma-separated values file\\
URL & Uniform Resource Locator/web address\\
Google API & Google’s Application Programming Interface\\
\bottomrule
\end{tabularx}

\end{table}

\begin{table}[!htbp]
\caption{\textbf{Table of Definitions}} \label{Table}

\begin{tabularx}{\textwidth}{p{3cm}X}
\toprule
\textbf{Term} & \textbf{Definition}\\
\midrule
Mosaic & McMaster University's online administrative information system.\\
MacID & A McMaster University student's login account.\\
Scrapy & A Web scraping Python library.\\
Crawler & Web crawler/spider that collects information from an html webpage.\\
XPATH Selector & Uses path expressions to select nodes/node-sets in an XML document. \\
Google Calendar & Google’s online free calendar connected to a gmail account.\\
Notepad ++ & Text editor.\\
%Term3 & Definition3\\
\bottomrule
\end{tabularx}

\end{table}	

\subsection{Overview of Document}
This application will be a reimplementation of the open source Chrome extension UMD Google Calendar Schedule Importer. The reimplementation will be modified to allow students from McMaster University to export their class schedule from Mosaic and import it into their Google Calendar.
\section{Plan}
	
\subsection{Software Description}
The software will parse Mosaic for a user's schedule and then export it to an array. This array will then be used to import their schedule into their Google calendar. The implementation will be completed in Python 3.6.

\subsection{Test Team}
The members of Team 0C are Cassandra Nicolak, Michelle Leung and Winnie Liang and are the test team for this project.
\subsection{Automated Testing Approach}

\subsection{Testing Tools}
The testing tools that will be used are PyTest and Pycharm. PyTest will be used to automate unit testing and Pycharm will be used for coverage checking and debugging.

\subsection{Testing Schedule}
See Gantt Chart at the following url: \color{blue}
\href{https://gitlab.cas.mcmaster.ca/liangw15/3XA3Project/blob/master/ProjectSchedule/Group12_Gantt03.pdf}{ Group12-Gantt03}

\color{black}

\section{System Test Description}
	
\subsection{Tests for Functional Requirements}

\subsubsection{Testing for User Input}
		
\paragraph{File Explorer}

\begin{enumerate}

\item{Test: FUI-01\\}

Type: Functional, Dynamic, Manual
					
Initial State:  File explorer that is used to select the html file of the schedule.
					
Input: Valid URL or absolute file path.

					
Output: The specified location of the html file will be saved at the URL required for the parsing component of the software.

					
How test will be performed: The function that acquires the URL input from the user will check if the XPATHs are present.

					
\item{testid2\\}

Type: Functional, Dynamic, Manual, Static etc.
					
Initial State: 
					
Input: 
					
Output: 
					
How test will be performed: 

\end{enumerate}

\subsubsection{Testing for Exporting from Mosaic}

\paragraph{Parsing}

\begin{enumerate}

\item{testid1\\}

Type: Functional, Dynamic, Manual, Static etc.
					
Initial State: 
					
Input: 
					
Output: 
					
How test will be performed: 

\item{testid2\\}

Type: Functional, Dynamic, Manual, Static etc.
					
Initial State: 
					
Input: 
					
Output: 
					
How test will be performed: 
\end{enumerate}

\subsubsection{Testing for Importing to Google Calendar}

\paragraph{Input Configuration}

\begin{enumerate}

\item{CV-01\\}
Test Time Format
Type: Functional, Dynamic
					
Initial State: None
					
Input: Time from CSV
					
Output: Time formatted according to RFC3339
					
How test will be performed: 
Time from the format that the parser outputs will be inputted into a function that converts it to RFC3339 format. The output of the function will be compared to the expected output. 

\item{CV-02\\}
Test Title Format
Type: Functional, Dynamic
					
Initial State: None
					
Input: From Parser
					
Output: Title for Event
					
How test will be performed: 
Data from parser will be input for a function that creates the title of the event. 
The output of the function will be compared to the expected output.

\item{CV-03\\}

Type: Functional, Dynamic
					
Initial State: None
					
Input: From Parser
					
Output: Description for Event
					
How test will be performed: 
Data from parser will be input for a function that creates the description of the event. 
The output of the function will be compared to the expected output.
\end{enumerate}


\subsubsection{Testing for Google API Handling}

\paragraph{Authentication and Authorization Test}
Must show that the class is able to connect to a user's Google account.
\begin{enumerate}

\item{Test: GC-0\\}

Type: Functional, Manual
					
Initial State: The client is not connected to their Google account. 
					
Input: Run function.
					
Output: Access to a user's account.
					
How test will be performed: Run function for authentication and authorization. 
User follows instructions. 
User will receive text confirming authentication and authorization.
\end{enumerate}

\paragraph{Information Retrieval and Upload Test}

Must show that the class for retrieving and exporting data to a Google calendar is able to do so for:
- Getting calendars
- Getting events in calendars
- Creating events in calendars
- Removing events in calendars

\begin{enumerate}

\item{Test: GC-01\\}

Type: Functional, Dynamic
					
Initial State: Calendars are present in Account
					
Input: None
Output: List of Calendars
					
How test will be performed: Run function. Check that a list of dictionaries; calendar ids with their title are returned.

\item{Test: GC-02\\}

Type: Functional, Dynamic
					
Initial State: Events are in specified calendar
					
Input: Calendar Id
Output: List of events
					
How test will be performed: How test will be performed: Run function. Check that a list of dictionaries; events ids and their properties are returned.

\item{Test: GC-03\\}

Type: Functional, Dynamic
					
Initial State:List of calendar events are saved
					
Input: Calendar Id, Event to be inserted.
Output: New list of events
					
How test will be performed: How test will be performed: Save current list of events from calendar, insert event into that calendar, get list of events from calendar, compare the new list to the old list to check that the event is inserted. 

\item{Test: GC-04\\}

Type: Functional, Dynamic
					
Initial State: List of calendar events are saved.

					
Input: Calendar Id, event to be deleted. 
Output: New list of events.
					
How test will be performed: Save current list of events from calendar, delete event from that calendar, get list of events from calendar, compare the new list to the old list to check that the event has been deleted.

\end{enumerate}


\subsection{Tests for Nonfunctional Requirements}

\subsubsection{Look and Feel}
		
\paragraph{Accessibility}

\begin{enumerate}

\item{testid1\\}

Type: 
					
Initial State: 
					
Input/Condition: 
					
Output/Result: 
					
How test will be performed: 
		
\item{testid2\\}

Type: Functional, Dynamic, Manual, Static etc.
					
Initial State: 
					
Input: 
					
Output: 
					
How test will be performed: 

\end{enumerate}

\subsubsection{Usability}

\begin{enumerate}

\item{US-00\\}

Type: Manual
					
Initial State: None
					
Input/Condition: None
					
Output/Result: User Survey Response
					
How test will be performed: Users will be asked to use the software and be given a survey asking about their experience. 

\end{enumerate}

\subsubsection{Performance}

\begin{enumerate}

\item{PF-00\\}

Type: Manual
					
Initial State: None
					
Input/Condition: None
					
Output/Result: User response
					
How test will be performed: 

Software will be manually tested and judged to see if the response time is reasonable. 

\end{enumerate}

\subsubsection{Maintainability}

\begin{enumerate}

\item{testid1\\}

Type: 
					
Initial State: 
					
Input/Condition: 
					
Output/Result: 
					
How test will be performed: 

\end{enumerate}

\subsubsection{Security}

\begin{enumerate}

\item{testid1\\}

Type: 
					
Initial State: 
					
Input/Condition: 
					
Output/Result: 
					
How test will be performed: 

\end{enumerate}

\subsection{Traceability Between Test Cases and Requirements}

\subsubsection{Functional Requirements}
\begin{tabular}{ |p{3cm}|p{8cm}|p{2cm}| }
\hline Requirement #  &Description/Summary& Test ID(s) \\
\toprule
FR01 & Notify user if unable to access Mosaic. & \\
FR02 & Notify user of information that will be imported into Google Calendar prior to proceeding. & \\
FR03 & Multiple uses from user. &\\
FR04 & Simple GUI & \\
FR05 & UI has ‘Help’ option. & \\
FR06 & Show user changes to existing Google Calendar prior to changing it. & \\
FR07 & Ask for confirmation to confirm that the listed changes are correct & \\
FR08 & Display a preview of the user’s timetable before importing. & \\
FR09 & Request permission to access user’s personal information in their Mosaic and Google account. & GC-00\\
FR10 & Have an option that allows a user to exit. & \\
\bottomrule
\end{tabular}

\subsubsection{Non-Functional Requirements}

\begin{tabular}{ |p{3cm}|p{8cm}|p{2cm}| }
\hline Requirement #  & Fit-Criterion/Summary & Test ID(s) \\
\toprule
NF01 & All information will be visible and not be dependant on colour.  & \\
NF02 & Perform the import in less than five steps. & \\
NF03 & Easy to use. & \\
NF04 & Application is easy to install. & \\
NF05 & All information should be at a basic English reading level.  & \\
NF06 & Status indicators. & \\
NF07 & Prompts to guide users step by step. & \\
NF08 & Run and use the application on a desktop computer or laptop.  & \\
NF09 & Respond to a user’s input in a reasonable amount of time (0.5 seconds). & \\
NF10 & Available on reliable site. & \\
NF11 & The programming language is supported on Windows and Linux.  & \\
NF12 & The source code can be accessed by the public.  & \\
NF13 & Current developers can be contacted by the public.  & \\
NF14 & The application will not have the ability to store user data. & \\
NF15 & Inoffensive display. & \\
NF16 & Adheres to relevant standards and laws. & \\
NF17 & Able to run the application with no harm to their health and safety.  & \\

\bottomrule
\end{tabular}

\section{Tests for Proof of Concept}

\subsection{Parser}
		
\paragraph{Title for Test}

\begin{enumerate}

\item{testid1\\}

Type: Functional, Dynamic, Manual, Static etc.
					
Initial State: 
					
Input: 
					
Output: 
					
How test will be performed: 
					
\item{testid2\\}

Type: Functional, Dynamic, Manual, Static etc.
					
Initial State: 
					
Input: 
					
Output: 
					
How test will be performed: 

\end{enumerate}

\subsection{Link to Google Calendar API}

\paragraph{Authentication and Authorization Test}
Must show that the class is able to connect to a user's Google account.
\begin{enumerate}

\item{Test: GC-00\\}

Type: Functional, Manual
					
Initial State: The client is not connected to their Google account. 
					
Input: Run function.
					
Output: Access to a user's account.
					
How test will be performed: Run function for authentication and authorization. 
User follows instructions. 
User will receive text confirming authentication and authorization.
\end{enumerate}

\paragraph{Information Retrieval and Upload Test}
Must show that the class for retrieving and exporting data to a Google calendar is able to do so for:
- Getting calendars
- Getting events in calendars
- Creating events in calendars
- Removing events in calendars

\begin{enumerate}

\item{Test: GC-01\\}

Type: Functional, Dynamic
					
Initial State: Calendars are present in Account
					
Input: None
Output: List of Calendars
					
How test will be performed: Run function. Check that a list of dictionaries; calendar ids with their title are returned.

\item{Test: GC-02\\}

Type: Functional, Dynamic
					
Initial State: Events are in specified calendar
					
Input: Calendar Id
Output: List of events
					
How test will be performed: How test will be performed: Run function. Check that a list of dictionaries; events ids and their properties are returned.

\item{Test: GC-03\\}

Type: Functional, Dynamic
					
Initial State:List of calendar events are saved
					
Input: Calendar Id, Event to be inserted.
Output: New list of events
					
How test will be performed: How test will be performed: Save current list of events from calendar, insert event into that calendar, get list of events from calendar, compare the new list to the old list to check that the event is inserted. 

\item{Test: GC-04\\}

Type: Functional, Dynamic
					
Initial State: List of calendar events are saved.

					
Input: Calendar Id, event to be deleted. 
Output: New list of events.
					
How test will be performed: Save current list of events from calendar, delete event from that calendar, get list of events from calendar, compare the new list to the old list to check that the event has been deleted.

\end{enumerate}


\section{Comparison to Existing Implementation}	
- Both parse an html webpage for information.

- Original uses JavaScript,Reimplementation uses Python 3.

- Different Universities

- Web Browser vs Desktop


\section{Unit Testing Plan}
Test cases will run on different operating systems, including Windows, and Linux.	
\subsection{Unit testing of internal functions}
Every internal function will be tested with the following cases, where applicable: 

\begin{enumerate}
\item A case where the function takes input it is expected to handle. 
\item All edge cases.
\item Cases for all run-time errors and exceptions. (Whitebox)
\item If the function receives input from a user, a case where it does not receive the expected input.
\end{enumerate}
	
\subsection{Unit testing of output files}		

\bibliographystyle{plainnat}

\bibliography{SRS}

\newpage

\section{Appendix}

This is where you can place additional information.

\subsection{Symbolic Parameters}

The definition of the test cases will call for SYMBOLIC\_CONSTANTS.
Their values are defined in this section for easy maintenance.

\subsection{Usability Survey Questions?}
\begin{enumerate}
\item How easy were the instructions to follow?

\item Were you confused at how to use the tool? If so, where and why?

\item Are there any other feature that you would like to see?

\end{enumerate}
This is a section that would be appropriate for some teams.

\end{document}
\end{document}