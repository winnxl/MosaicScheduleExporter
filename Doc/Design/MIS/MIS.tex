\documentclass[12pt, titlepage]{article}

\usepackage{fullpage}
\usepackage[round]{natbib}
\usepackage{multirow}
\usepackage{booktabs}
\usepackage{tabularx}
\usepackage{graphicx}
\usepackage{float}
\usepackage{hyperref}
\usepackage[normalem]{ulem}
\hypersetup{
    colorlinks,
    citecolor=black,
    filecolor=black,
    linkcolor=red,
    urlcolor=blue
}
\usepackage[round]{natbib}

\newcounter{acnum}
\newcommand{\actheacnum}{AC\theacnum}
\newcommand{\acref}[1]{AC\ref{#1}}

\newcounter{ucnum}
\newcommand{\uctheucnum}{UC\theucnum}
\newcommand{\uref}[1]{UC\ref{#1}}

\newcounter{mnum}
\newcommand{\mthemnum}{M\themnum}
\newcommand{\mref}[1]{M\ref{#1}}

\title{SE 3XA3: Module Internal Specification\\MAC Schedule Importer}

\author{Team 12, 0C
		\\ Cassandra Nicolak, nicolace
		\\ Michelle Leung, leungm16
		\\ Winnie Liang, liangw15
}

\date{\today}

%\input{../../Comments}

\begin{document}

\maketitle

\pagenumbering{roman}
\tableofcontents
\listoftables
\listoffigures

\begin{table}[bp]
\caption{\bf Revision History}
\begin{tabularx}{\textwidth}{p{3cm}p{2cm}X}
\toprule {\bf Date} & {\bf Version} & {\bf Notes}\\
\midrule
2018-11-07 & 1.0 & Rough Draft\\
2018-12-04 & 1.1 & Updated renames and added new methods.\\
\bottomrule
\end{tabularx}
\end{table}

\newpage

\pagenumbering{arabic}

\section{Introduction}
\hspace{5mm} The purpose of this module internal specification document is to provide a complete description of the specifications to design the MAC Schedule Importer. The project is a  redesign ofthe open-source Chrome extension, UMD Google Calendar Schedule Importer, which imports the class schedule for students at the University of Maryland into Google Calendar. The reimplementation will be modified to allow students from McMaster University to import their schedules from Mosaic through a Desktop application.

\section{Module Hierarchy} \label{SecMH}

This section provides an overview of the module design. Modules are summarized
in a hierarchy decomposed by secrets in Table \ref{TblMH}. The modules listed
below, which are leaves in the hierarchy tree, are the modules that will
actually be implemented.

%\begin{description}
%\item [\refstepcounter{mnum} \mthemnum \label{mHH}:] Hardware-Hiding %Module
%\item ...
%\end{description}


\begin{table}[h!]
\centering
\begin{tabular}{p{0.3\textwidth} p{0.6\textwidth}}
\toprule
\textbf{Level 1} & \textbf{Level 2}\\
\midrule

{Hardware-Hiding Module} & ~ \\
\midrule

\multirow{3}{0.3\textwidth}{Behaviour-Hiding Module} & converter\\
& connector\\
& guiClient\\
\midrule

%{Software Decision Module} & ~ \\
\multirow{1}{0.3\textwidth}{Software Decision Module} & parseMosaic\\
%& ?\\
%& ?\\
\bottomrule

\end{tabular}
\caption{Module Hierarchy}
\label{TblMH}
\end{table}
\newpage

\section{MIS of ParseMosaic} \label{SecM2}
\subsection{Uses}
scrapy, subprocess
\subsection{Interface Syntax}
\subsubsection{Exported Access Programs}
\begin{tabular}[pos]{|c|c|c|c|}
	\hline
	\textbf{Name}& \textbf{In} & \textbf{Out} & \textbf{Exceptions} \\ \hline
	parse & response: TextResponse & \sout{dataList} \color{blue}data\_list\color{black}: list of string tuples & -\\ \hline %!! What's the type for response?
	%close & - & GUI & -\\ \hline
	\sout{runMe}  \color{blue}run\_me\color{black} & passed\_url: url (str) & GUI & -\\ \hline
			
\end{tabular}\\
$*$Note: Exceptions are still in development.

\subsection{Interface Semantics}
\subsubsection{State Variables}
\sout{dataList} \color{blue}data\_list\color{black} : list of string tuples
\subsubsection{Environmental Variables}
process : CrawlerSpiderProcess() from the Scrapy Library %process? name?
\subsubsection{State Invariant}
$0 \le |$\sout{dataList} \color{blue}data\_list\color{black}$|$
\subsubsection{Assumptions}
parse() is called before \sout{runMe()} \color{blue}run\_me()\color{black}.
\subsubsection{Access Program Semantics}
parse(response)
\begin{itemize}
    \item transition: \sout{dataList} \color{blue}data\_list\color{black} $:= $ modify \sout{dataList} \color{blue}data\_list \color{black} so that it uses the Scrapy library to parse data from \emph{response} and stores a list of string tuples with the tuple containing (course name, component, schedule, location, dates) of each course.
    \item exception : None
\end{itemize}
\sout{runMe()} \color{blue}run\_me()\color{black}
\begin{itemize}
    \item output $:= $ \sout{dataList} \color{blue}data\_list\color{black} 
    \item exception: None
\end{itemize}

\newpage

\section{MIS of Converter} \label{SecM1}
\subsection{Uses}
parseMosaic

\subsection{Interface Syntax}
\subsubsection{Exported Access Programs}
\begin{tabular}[pos]{|c|c|c|c|}
	\hline
	\textbf{Name}& \textbf{In} &  \textbf{Out} & \textbf{Exceptions} \\ \hline
	extract\_date & input: str & start: str, end: str & -\\
	\hline
	to\_military & input: str & mil: str & -\\
	\hline
	extract\_weekdays & input: str & weekdays: str & - \\
	\hline 
	rfc\_output & date\_str, time\_str: str & start\_date\_time, end\_date\_time: str & - \\
	\hline 
	convert & input: list of string tuples & output: list of dictionaries & - \\
	\hline 
\end{tabular}
*Note: Exceptions are still in development.
\subsection{Interface Semantics}
\subsubsection{State Variables}
None
%fileUrl: str - The url of the user's Mosaic schedule.\\
%ret: list of tuples of str - The information parsed from Mosaic.
\subsubsection{Environmental Variables}
None
\subsubsection{State Invariant}
None
\subsubsection{Assumptions}
None, unless stated in the access program.
\subsubsection{Access Program Semantics}
extract\_date(input)
\begin{itemize}
    \item output: start $:= (\exists i | i \in input \land input[i] = $'-' : $ start = input[0..i-1]) $ 
    \item output: end $:= $
    $(\exists i | i \in input \land input[i] = $'-' : $ end = input[i+1..|input|]) $
    \item exception: None
\end{itemize}
to\_military(input)
\begin{itemize}
    \item output: mil := returns a string of  the military time given a 12-hour time input string.
    \item exception: None
\end{itemize}
extract\_weeekdays(input)
\begin{itemize}
    \item output: weekdays := given a string containing weekdays, ex. "MoTWeThFr", it returns a string with capitalized string with commas between the weekdays, ex."MO,TU,WE,TH,FR" 
    \item exception: None
\end{itemize}
rfc\_output(date\_str, time\_str)
\begin{itemize}
    \item output: a start and end dateTime in RFC 2232 format, and a rrule in Rfc 5545 format
    \item exception: None
\end{itemize}
convert(input)
\begin{itemize}
    \item output: a list of dictionaries containing calendar event parameters
    \item exception: None
\end{itemize}

\newpage
\section{MIS of Connector} \label{SecM4}
\subsubsection{Uses}
sys, os, googleapiclient, socket, oauth2client, httplib2
\subsection{Interface Syntax}
\subsubsection{Exported Access Programs}
\begin{tabular}[pos]{|p{3cm}|p{1cm}|p{2cm}|p{10cm}|}
	\hline
	\textbf{Name}& \textbf{In} & \textbf{Out} & \textbf{Exceptions} \\ \hline
	login & - & GUI & ServerNotFoundError\\ \hline
	logout & - & GUI & -\\ \hline 
	check\_perms & - & bool & -\\ \hline
	create\_cal & name: str & Google Calendars & AccessTokenRefreshError, ServerNotFound Error, gaierror, ConnectionResetError \\ \hline
	insert\_events & - & True, None & AccessTokenRefreshError, ServerNotFound Error, gaierror, ConnectionResetError\\ \hline
	get\_num\_events & - & $|$event\_list$|$: int & AccessTokenRefreshError, ServerNotFound Error, gaierror, ConnectionResetError \\ \hline
	check\_insertion & - & bool & - \\ \hline
	remove\_new\_cal & - & bool & AccessTokenRefreshError, ServerNotFound Error, gaierror, ConnectionResetError\\ \hline
	push\_to\_schedule & - & bool & -\\ \hline
	
\end{tabular}

\subsection{Interface Semantics}
\subsubsection{State Variables}
service: Object\\
cal\_id: string \\
bodies: dictionary \\
\subsubsection{Environmental Variables}
None
\subsubsection{Assumptions}
None, unless noted in the access programs
\subsubsection{Access Program Semantics}
login 
\begin{itemize}
    \item transition: service $:=$ Object
    \item exception: None
\end{itemize}
logout
\begin{itemize}
    \item transition: service $:=$ None
    \item exception: None
\end{itemize}
check\_perms
\begin{itemize}
    \item output: Bool, true if permissions were granted, False otherwise.
    \item exception: None
\end{itemize}
create\_cal
\begin{itemize}
    \item transition: cal\_id $:=$ new calendar id
    \item exception: client.AccessTokenRefreshError, when access token fails to refresh.
    \item exception: gaierror, internet fails
    \item exception: ConnectionResetError, internet fails
    \item exception: ServerNotFoundError, internet fails
\end{itemize}
insert\_events
\begin{itemize}
    \item output: inserts event into google calendar
    \item exception: client.AccessTokenRefreshError, when access token fails to refresh.
    \item exception: gaierror, internet fails
    \item exception: ConnectionResetError, internet fails
    \item exception: ServerNotFoundError, internet fails
\end{itemize}
get\_num\_events
\begin{itemize}
    \item output: number of events in calendar: cal\_id
    \item exception: client.AccessTokenRefreshError, when access token fails to refresh.
    \item exception: gaierror, internet fails
    \item exception: ConnectionResetError, internet fails
    \item exception: ServerNotFoundError, internet fails
\end{itemize}
check\_insertion
\begin{itemize}
    \item output: True if number of events in Google calendars matches the number of event parameters sent to Google. False otherwise
\end{itemize}
remove\_new\_cal
\begin{itemize}
    \item output: removal of calendar: cal\_id
    \item exception: client.AccessTokenRefreshError, when access token fails to refresh. 
    \item exception: gaierror, internet fails
    \item exception: ConnectionResetError, internet fails
    \item exception: ServerNotFoundError, internet fails
\end{itemize}
push\_to\_schedule
\begin{itemize}
    \item output: Pushes events to Google calendars. Returns True if successful. False otherwise.
\end{itemize}

\newpage
\color{blue}
\section{MIS of Setup} \label{SecM4}
\subsection{Uses}
sys, os, cx\_freeze

\subsection{Interface Syntax}
\subsubsection{Exported Access Programs}
Not applicable.
\subsection{Interface Semantics}
\subsubsection{State Variables}
build\_exe\_options
\subsubsection{Environmental Variables}
os.environ['TCL\_LIBRARY']\\
os.environ['TK\_LIBRARY']
\subsubsection{State Invariant}
None
\subsubsection{Assumptions}
None, unless stated in the access program.
\subsubsection{Access Program Semantics}
This module is a configuration file for cx\_freeze, a software that converts Python programs into executable applications. The module consists of import statements that imports all relevant libraries and a setup function of cx\_freeze. The function contains an input file parameter with the input Python program guiClient.py and build options that include the packages that is required to build the program.
\color{black}

\newpage
\section{MIS of guiClient} \label{SecM5}
\subsubsection{Uses}
PySimpleGUI, parseMosaic, connector, converter, urllib, webbrowser


\subsection{Interface Syntax}
\subsubsection{Exported Access Programs}
\begin{tabular}[pos]{|c|c|c|c|}
	\hline
	\textbf{Name}& \textbf{In} & \textbf{Out} & \textbf{Exceptions} \\ \hline
    \sout{convertURL} &  &  &  \\ 
    \color{blue}convert\_url\color{black} & str & str & - \\ \hline		
	\sout{parseMosaic} &  & \sout{fetchedList: list of tuple of str} &  \\  
	\color{blue}parse\_mosaic\color{black} & str &  \color{blue}fetched\_list\color{black}: list of tuple of str & -\\ \hline
	\color{blue}set\_list & \color{blue}parsed\_list: list of tuple of str & \color{blue}fetched\_list: list of tuple of str & \color{blue}-\color{black} \\ \hline
    \sout{printSched} & \sout{fetchedList: list of tuple of str} &  &  \\ 
    \color{blue}print\_sched\color{black} & \color{blue}fetched\_list\color{black}: list of tuple of str & str & - \\ \hline	
	fetch & str & str & - \\ \hline
	conn & - & Object & - \\ \hline	
	login & - & bool & - \\ \hline
	logout & - & \color{blue}bool\color{black} & \color{blue}AttributeError\color{black} \\ \hline	
    \sout{pushSchedule} &   &  & \\ 
    \color{blue}push\_schedule\color{black} & - & bool & - \\ \hline
	\color{blue}fetch\_button & \color{blue}- & \color{blue}- & \color{blue}-\color{black} \\ \hline
	\color{blue}fetch\_popup & \color{blue}- & \color{blue}bool & \color{blue}-\color{black} \\ \hline	
	\color{blue}login\_button & \color{blue}- & \color{blue}- & \color{blue}-\color{black} \\ \hline
	\color{blue}import\_button & \color{blue}- & \color{blue}- & \color{blue}AttributeError\color{black} \\ \hline
% \color{blue}data\_list\color{black}			
\end{tabular}

\subsection{Interface Semantics}
\subsubsection{State Variables}
\sout{fetchFLG} \color{blue}\_fetch\_flg\color{black}: bool\\
\sout{fetchedList} \color{blue}\_fetched\_list\color{black}: list of string tuples\\
\sout{googleConn} \color{blue}\_google\_conn\color{black}: None
\subsubsection{Environmental Variables}
layout: list of lists of type GUI\\
window: GUI\\
menu\_def: list of lists of str
\subsubsection{State Invariant}
$|$\sout{fetchedList} \color{blue}\_fetched\_list\color{black}$| \ge 0$
\subsubsection{Assumptions}
None, unless stated by the access programs.
\subsubsection{Access Program Semantics}
\sout{convertURL(userInput)} \color{blue}convert\_url(user\_input)\color{black} 
\begin{itemize}
    \item output: \sout{userURL} \color{blue}user\_url\color{black}  $:=$ given a url from a path name (str), it returns the absolute file path (str).
    \item exception: None
\end{itemize}
\sout{parseMosaic(url)} \color{blue}parse\_mosaic(url)\color{black}
\begin{itemize}
    \item output: url:= Parses the html document from \emph{url} and displays the \emph{url} on the GUI.
    \item exception: None
\end{itemize}
\color{blue}set\_list()\color{black}
\begin{itemize}
    \color{blue}\item output: url:= Sets \emph{\_fetched\_list} to \emph{parsed\_list}.\color{black}
    \color{blue}\item exception: None\color{black}
\end{itemize}
\sout{printSched(fetchList)} \color{blue}print\_sched(fetch\_list)\color{black}
\begin{itemize}
    \item output: out:= Converts \sout{fetchList} \color{blue}fetch\_list \color{black} into the following format of str: "Course, Type, When, Location, Start/End Dates"
\end{itemize}
fetch(url)
\begin{itemize}
    \item output: out:= Parses url using\sout{parseMosaic()} \color{blue}parse\_mosaic() \color{black} function and returns the schedule using \sout{printSched()} \color{blue}print\_sched()\color{black}.
    \item exception: None
\end{itemize}
conn()
\begin{itemize}
    \item transition: Converts the output of \sout{parseMosaic} \color{blue}parse\_mosaic \color{black} to Google API inputs.
    \item exception: None
\end{itemize}
login()
\begin{itemize}
    \item output: out:= Creates a link to google's api service when the user logs into their Google account. Returns true if there is a service, otherwise false.
    \item exception: None
\end{itemize}
logout()
\begin{itemize}
    \item \sout{transition: deletes access key to user's account. }
    \item \color{blue}output: out:= Deletes access key to user's account. Returns true if this is successful, otherwise false.\color{black}
    \item exception: \sout{None} \color{blue}AttributeError\color{black}
\end{itemize}
\sout{pushSchedule()} \color{blue}push\_schedule()\color{black}
\begin{itemize}
    \item output: out:= Uploads event items to a Google Calendar. Returns true if the import is successful and false otherwise.
\end{itemize}
\color{blue}fetch\_button()\color{black}
\begin{itemize}
    \color{blue}\item output: url:= Updates the \emph{tbxSchedule} textbox with a str.\color{black}
    \color{blue}\item exception: None\color{black}
\end{itemize}
\color{blue}fetch\_popup()\color{black}
\begin{itemize}
    \color{blue}\item output: url:= Displays a pop-up window to the user. Returns true if user selects 'Yes', otherwise false.\color{black}
    \color{blue}\item exception: None\color{black}
\end{itemize}
\color{blue}login\_button()\color{black}
\begin{itemize}
    \color{blue}\item transition: url:= Opens a new connection with \emph{conn()}.\color{black}
    \color{blue}\item output: url:= Updates the \emph{tbxLogin} textbox with a str.\color{black}
    \color{blue}\item exception: None\color{black}
\end{itemize}
\color{blue}import\_button()\color{black}
\begin{itemize}
    \color{blue}\item transition: url:= Updates the \emph{tbxImport} textbox with a str.\color{black}
    \color{blue}\item exception: AttributeError\color{black}
\end{itemize}

%\begin{figure}[H]
%\centering
%\includegraphics[width=0.7\textwidth]{UsesHierarchy.png}
%\caption{Use hierarchy among modules}
%\label{FigUH}
%\end{figure}

%\section*{References}

\bibliographystyle {plainnat}
\bibliography {MG}

\end{document}
\\
